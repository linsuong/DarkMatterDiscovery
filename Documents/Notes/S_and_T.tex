\documentclass[a4paper,12pt]{article}
\usepackage{amsmath,amssymb}
\usepackage{hyperref}
 
\begin{document}
 
\title{Electroweak Precision Tests: The $S$, $T$, and $U$ Parameters}
\author{}
\date{}
\maketitle
 
\section{Introduction}
 
The $S$, $T$, and $U$ parameters, also known as oblique parameters, are used in electroweak precision tests to describe deviations from the Standard Model (SM) due to new physics beyond it. These parameters, introduced by Peskin and Takeuchi, quantify modifications to vacuum polarisation corrections of electroweak gauge bosons without directly affecting the fermion couplings.
 
\section{Definitions of $S$, $T$, and $U$ Parameters}
 
The oblique parameters are defined as follows:
 
\subsection{$S$ Parameter}
 
The $S$ parameter measures new physics contributions to the difference between neutral and charged current processes:
\begin{equation}
S = \frac{4s_W^2c_W^2}{\alpha} \left[ \Pi'_{ZZ}(0) - \Pi'_{\gamma\gamma}(0) - 2\frac{c_W^2 - s_W^2}{c_Ws_W} \Pi'_{\gamma Z}(0) \right].
\end{equation}
It is sensitive to modifications in the running of electroweak couplings due to high-energy new physics.
 
\subsection{$T$ Parameter}
 
The $T$ parameter quantifies isospin-breaking effects due to new physics contributions that modify the mass splitting between the $W$ and $Z$ bosons:
\begin{equation}
T = \frac{1}{\alpha} \left[ \frac{\Pi_{WW}(0)}{M_W^2} - \frac{\Pi_{ZZ}(0)}{M_Z^2} \right].
\end{equation}
It is related to the electroweak $\rho$-parameter:
\begin{equation}
\rho = \frac{M_W^2}{M_Z^2 \cos^2 \theta_W} = 1 + \alpha T.
\end{equation}
New physics violating custodial symmetry, such as extra Higgs doublets or heavy fermion mass splittings, significantly affects $T$.
 
\subsection{$U$ Parameter}
 
The $U$ parameter measures differences between new physics effects on the $W$ and $Z$ bosons, analogous to $S$:
\begin{equation}
U = \frac{4s_W^2}{\alpha} \left[ \Pi'_{WW}(0) - \Pi'_{ZZ}(0) \right].
\end{equation}
It is typically less constrained than $S$ and $T$ as many new physics models do not introduce significant $U$-dependent corrections.
 
\section{Experimental Constraints and Applications}
 
The $S$, $T$, and $U$ parameters provide a model-independent framework for testing new physics effects in electroweak interactions. Constraints from precision measurements at LEP, SLC, Tevatron, and the LHC put stringent bounds on these parameters.
 
Several beyond-the-Standard-Model (BSM) scenarios predict specific shifts in $S$, $T$, and $U$, making them useful for distinguishing different new physics models, such as:
\begin{itemize}
    \item Technicolor models
    \item Extra-dimensional models
    \item Heavy vector-like fermion models
\end{itemize}
 
For recent experimental constraints, refer to the LEP Electroweak Working Group: \url{https://lepewwg.web.cern.ch/}.
 
\section{Charged and Neutral Currents in the Standard Model}
 
\subsection{Charged Currents}
 
Charged current interactions change the charge of a particle via the exchange of a $W^\pm$ boson. An example is beta decay:
\begin{equation}
n \to p + e^- + \bar{\nu}_e.
\end{equation}
The charged current interaction in the SM Lagrangian is given by:
\begin{equation}
\mathcal{L}_{CC} = \frac{g}{\sqrt{2}} \left( \bar{u} \gamma^\mu P_L d + \bar{\nu}_e \gamma^\mu P_L e \right) W^+_\mu + \text{h.c.}
\end{equation}
where $g$ is the weak coupling and $P_L = \frac{1}{2} (1 - \gamma^5)$ projects onto left-handed fields.
 
\subsection{Neutral Currents}
 
Neutral current interactions do not change the charge of a particle and are mediated by the $Z$ boson or photon. An example is neutrino-electron scattering:
\begin{equation}
\nu_e + e^- \to \nu_e + e^-.
\end{equation}
The neutral current interaction term in the SM Lagrangian is:
\begin{equation}
\mathcal{L}_{NC} = \frac{g}{2c_W} \sum_f \bar{f} \gamma^\mu (g_V - g_A \gamma^5) f Z_\mu.
\end{equation}
The vector and axial-vector couplings $g_V$ and $g_A$ depend on the weak mixing angle $\theta_W$.
 
\section{References}
 
\begin{enumerate}
    \item M. E. Peskin, T. Takeuchi, \textit{Phys. Rev. Lett.} \textbf{65}, 964 (1990).
    \item M. E. Peskin, T. Takeuchi, \textit{Phys. Rev. D} \textbf{46}, 381 (1992).
    \item A. Pich, \textit{Precision Electroweak Tests and Physics Beyond the Standard Model}, \texttt{arXiv:hep-ph/9505231}.
    \item G. Altarelli, R. Barbieri, F. Caravaglios, \textit{Int. J. Mod. Phys. A} \textbf{13}, 1031 (1998), \texttt{arXiv:hep-ph/9712368}.
    \item J. Erler, P. Langacker, \textit{Phys. Rev. Lett.} \textbf{105}, 031801 (2010), \texttt{arXiv:1003.3211}.
    \item LEP Electroweak Working Group: \url{https://lepewwg.web.cern.ch/}.
\end{enumerate}
 
\end{document}