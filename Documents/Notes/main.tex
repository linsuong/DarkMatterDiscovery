%--------------------
% Packages
% -------------------
\documentclass[11pt,a4paper]{article}
\usepackage[utf8x]{inputenc}
\usepackage[T1]{fontenc}
%\usepackage{gentium}
\usepackage{mathptmx} % Use Times Font
\usepackage{esvect}
\usepackage{mathrsfs}


\usepackage[pdftex]{graphicx} % Required for including pictures
\usepackage[swedish]{babel} % Swedish translations
\usepackage[pdftex,linkcolor=black,pdfborder={0 0 0}]{hyperref} % Format links for pdf
\usepackage{calc} % To reset the counter in the document after title page
\usepackage{enumitem} % Includes lists

\frenchspacing % No double spacing between sentences
\linespread{1.2} % Set linespace
\usepackage[a4paper, lmargin=0.1666\paperwidth, rmargin=0.1666\paperwidth, tmargin=0.1111\paperheight, bmargin=0.1111\paperheight]{geometry} %margins
%\usepackage{parskip}

\usepackage[all]{nowidow} % Tries to remove widows
\usepackage[protrusion=true,expansion=true]{microtype} % Improves typography, load after fontpackage is selected

\usepackage{lipsum} % Used for inserting dummy 'Lorem ipsum' text into the template


%-----------------------
% Set pdf information and add title, fill in the fields
%-----------------------
\hypersetup{ 	
pdfsubject = {The Discovery of Dark Matter - NOTES},
pdftitle = {},
pdfauthor = {}
}

%-----------------------
% Begin document
%-----------------------
\begin{document} %All text i dokumentet hamnar mellan dessa taggar, allt ovanför är formatering av dokumentet

\section{On Particle Physics}
\subsection{Standard Model}
\subsection{Cross Sectional Area}Propagator:
\begin{equation}
    \frac{1}{s} = \frac{1}{q_0^2 - |q|^2} = \frac{1}{{E^{TOT}_{CoM}}^2}
\end{equation}

\subsection{Interactions}
\subsubsection{Electromagnetic}
Intermediate particle (virtual particle): EM wave, $\gamma$
Conservation Laws:
\begin{itemize}
    \item E, $\vec{p}$ at each vertex
    \item Charge, Q
    \item Lepton number ($L_{e^-} = +1$, $L_{e^+} = -1$)
    \item Flavour
\end{itemize}

\subsubsection{Strong}
Intermediate particle (virtual particle): Gluon, g
\begin{itemize}
    \item E, $\vec{p}$ at each vertex
    \item Flavour
\end{itemize}
Note: No electric charges in play, and leptons do not interact strongly.

\subsubsection{Weak (Neutral)}
Intermediate particle (virtual particle): Z boson, $Z^0$
Similar conservation laws to EM:
\begin{itemize}
    \item E, $\vec{p}$ at each vertex
    \item Charge, Q
    \item Lepton number
    \item Flavour
\end{itemize}

\subsubsection{Weak (Charged)}
Intermediate particle (virtual particle): W boson, $W^+$ or $W^-$
Conservation laws:
\begin{itemize}
    \item E, $\vec{p}$ at each vertex
    \item Charge, Q
    \item Lepton number
\end{itemize}

\subsection{Feynman Diagrams}
{
\unitlength=1.0 pt
\SetScale{1.0}
\SetWidth{0.7}      % line    size control
\scriptsize    %  letter  size control
{} \qquad\allowbreak
%  diagram # 1
\begin{picture}(96,38)(0,0)
\ArrowLine(12.0,35.0)(36.0,23.0) 
\Text(12.0,35.0)[r]{$e$}
\ArrowLine(36.0,23.0)(12.0,11.0) 
\Text(12.0,11.0)[r]{$E$}
\DashLine(36.0,23.0)(60.0,23.0){3.0} 
\Text(49.0,24.0)[b]{$A$}
\ArrowLine(60.0,23.0)(84.0,35.0) 
\Text(84.0,35.0)[l]{$m$}
\ArrowLine(84.0,11.0)(60.0,23.0) 
\Text(84.0,11.0)[l]{$M$}
\end{picture} \ 
{} \qquad\allowbreak
%  diagram # 2
\begin{picture}(96,38)(0,0)
\ArrowLine(12.0,35.0)(36.0,23.0) 
\Text(12.0,35.0)[r]{$e$}
\ArrowLine(36.0,23.0)(12.0,11.0) 
\Text(12.0,11.0)[r]{$E$}
\DashLine(36.0,23.0)(60.0,23.0){3.0} 
\Text(49.0,24.0)[b]{$Z$}
\ArrowLine(60.0,23.0)(84.0,35.0) 
\Text(84.0,35.0)[l]{$m$}
\ArrowLine(84.0,11.0)(60.0,23.0) 
\Text(84.0,11.0)[l]{$M$}
\end{picture} \ 
}

\begin{figure}
    \Fig\eE-nM.png
\end{figure}
\subsection{Feynman Rules}
To obtain the amplitude of an interaction, there are some rules used. For an EM interaction:
\begin{itemize}
    \item Vertex is proportional to the charge
    \item Conservation of E and $\vec{p}$ at each vertex
    \item The propagator: 
        \begin{equation}
            D(q_o, \vec{q}) = \frac{1}{q_0^2 - |\vec{q}|^2 - M + i\Gamma M}
        \end{equation}
        Where $q_0 = \frac{(E^1 _f - E^1 _i)}{c} = \frac{(E^2 _f - E^2 _i)}{c}$
        and $\vec{q} = \vec{p}^1_f - \vec{p}^1_i = \vec{p}^2_f - \vec{p}^2_i$
\end{itemize}
And so the amplitude of the interaction, $\mathscr{A}$ is proportional to:
\begin{equation}
    \frac{e_1 e_2}{q_0^2 - |\vec{q}|^2} = \frac{e_1 e_2}{\frac{(E^1 _f - E^1 _i)}{c} - \vec{p}^1_f - \vec{p}^1_i}
\end{equation}

And we can say that $\mathscr{A} \propto \frac{e^2}{s}$

\section{Detection of Dark Matter}
\subsection{Direct}
Underground Dirac detection
\subsection{Indirect}
Collider search, where:
\begin{itemize}
    \item $E = mc^2$, can derive mass of DM from energy of collider
    \item A visible particle is recoiled by an invisible particle
    \item Missing momenta can deduce mass of Dark matter
\end{itemize}
\section{}

\section{CalcHEP Notes}


\end{document}

